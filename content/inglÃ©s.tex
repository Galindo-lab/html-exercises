
In the temperate and tropical regions where it
appears that hominids evolved into human
beings, the principal food of the species was
vegetable. Sixty---five to eighty percent of what
human beings ate in those regions in Paleolithic,
Neolithic, and prehistoric times was gathered;
only in the extreme Arctic was meat the staple
food. The mammoth hunters spectacularly
occupy the cave wall and the mind, but what we
actually did to stay alive and fat was gather seeds,
roots, sprouts, shoots, leaves, nuts, berries, fruits,
and grains, adding bugs and mollusks and netting
or snaring birds, fish, rats, rabbits, and other
tuskless small fry to up the protein. And we didn’t
even work hard at it --- much less hard than
peasants slaving in somebody else’s field after
agriculture was invented, much less hard than
paid workers since civilization was invented.
The average prehistoric person could make a
nice living in about a fifteen---hour work week.

Fifteen hours a week for subsistence leaves a
lot of time for other things. So much time that
maybe the restless ones who didn’t have a baby
around to enliven their life, or skill in making or
cooking or singing, or very interesting thoughts
to think, decided to slope off and hunt
mammoths. The skillful hunters would come
staggering back with a load of meat, a lot of ivory,
and a story. It wasn’t the meat that made the
difference. It was the story.

It is hard to tell a really gripping tale of how I
wrestled a wild-oat seed from its husk, and then
another, and then another, and then another, and
then another, and then I scratched my gnat bites,
and Ool said something funny, and we went to
the creek and got a drink and watched newts for
a while, and then I found another patch of oats....
No, it does not compare, it cannot compete with
how I thrust my spear deep into the titanic hairy
flank while Oob, impaled on one huge sweeping
tusk, writhed screaming, and blood sprouted
everywhere in crimson torrents, and Boob was
crushed to jelly when the mammoth fell on him
as I shot my unerring arrow straight through eye
to brain.

That story not only has Action, it has a Hero.
Heroes are powerful. Before you know it, the
men and women in the wild-oat patch and their
kids and the skills of makers and the thoughts of
the thoughtful and the songs of the singers are all
part of it, have all been pressed into service in the
tale of the Hero. But it isn’t their story. It’s his.

When she was planning the book that ended
up as Three Guineas, Virginia Woolf wrote a
heading in her notebook, ''Glossary''; she had
thought of reinventing English according to her
new plan, in order to tell a different story. One of
the entries in this glossary is heroism, defined as
''botulism.'' And hero, in Woolf’s dictionary, is
''bottle.'' The hero as bottle, a stringent
reevaluation. I now propose the bottle as hero.

Not just the bottle of gin or wine, but bottle
in its older sense of container in general, a thing
that holds something else.

If you haven’t got something to put it in, food
will escape you even something as
uncombative and unresourceful as an oat. You
put as many as you can into your stomach while
they are handy, that being the primary container;
but what about tomorrow morning when you
wake up and it’s cold and raining and wouldn’t it
be good to have just a few handfuls of oats to
chew on and give little Oom to make her shut up,
but how do you get more than one stomachful
and one handful home? So you get up and go to
the damned soggy oat patch in the rain, and
wouldn’t it be a good thing if you had something
to put Baby Oo Oo in so that you could pick the
oats with both hands? A leaf a gourd shell a net a
bag a sling a sack a bottle a pot a box a container.
A holder. A recipient. 

\begin{displayquote}
    \interlinepenalty=10000
    The first cultural device was probably a
    recipient... Many theorizers feel that the earliest
    cultural inventions must have been a container
    to hold gathered products and some kind of sling
    or net carrier.
\end{displayquote}

So says Elizabeth Fisher in \textit{Women’s
Creation} (McGraw-Hill, 1975). But no, this
cannot be. Where is that wonderful, big, long,
hard thing, a bone, I believe, that the Ape Man
first bashed somebody in the movie and then,
grunting with ecstasy at having achieved the first
proper murder, flung up into the sky, and
whirling there it became a space ship thrusting its
way into the cosmos to fertilize it and produce at
the end of the movie a lovely fetus, a boy of
course, drifting around the Milky Way without
(oddly enough) any womb, any matrix at all? I
don’t know. I don’t even care. I’m not telling that
story. We’ve heard it, we’ve all heard about all
the sticks and spears and swords, the things to
bash and poke and hit with, the long, hard things,
but we have not heard about the thing to put
things in, the container for the thing contained.
That is a new story. That is news.

And yet old. Before ---once you think about
it, surely long before--- the weapon, a late,
luxurious, superfluous tool; long before the
useful knife and ax; right along with the
indispensable whacker, grinder, and digger --- for
what’s the use of digging up a lot of potatoes if
you have nothing to lug the ones you can’t eat
home in --- with or before the tool that forces
energy outward, we made the tool that brings
energy home. It makes sense to me. I am an
adherent of what Fisher calls the Carrier Bag
Theory of human evolution.

This theory not only explains large areas of
theoretical obscurity and avoids large areas of
theoretical nonsense (inhabited largely by tigers,
foxes, and other highly territorial mammals); it
also grounds me, personally, in human culture in
a way I never felt grounded before. So long as
culture was explained as originating from and
elaborating upon the use of long, hard objects for
sticking, bashing, and killing, I never thought that
I had, or wanted, any particular share in it.
(''What Freud mistook for her lack of civilization
is woman’s lack of \textit{loyalty} to civilization,'' Lillian
Smith observed.) The society, the civilization
they were talking about, these theoreticians, was
evidently theirs; they owned it, they liked it; they
were human, fully human, bashing, sticking,
thrusting, killing. Wanting to be human too, I
sought for evidence that I was; but if that’s what
it took, to make a weapon and kill with it, then
evidently I was either extremely defective as a
human being, or not human at all.

That’s right, they said. What you are is a
woman. Possibly not human at all, certainly
defective. Now be quiet while we go on telling
the Story of the Ascent of Man the Hero.

Go on, say I, wandering off towards the wild
oats, with Oo Oo in the sling and little Oom
carrying the basket. You just go on telling how
the mammoth fell on Boob and how Cain fell on
Abel and how the bomb fell on Nagasaki and how
the burning jelly fell on the villagers and how the
missiles will fall on the Evil Empire, and all the
other steps in the Ascent of Man.

If it is a human thing to do to put something
you want, because it’s useful, edible, or beautiful,
into a bag, or a basket, or a bit of rolled bark or
leaf, or a net woven of your own hair, or what
have you, and then take it home with you, home
being another, larger kind of pouch or bag, a
container for people, and then later on you take
it out and eat it or share it or store it up for winter
in a solider container or put it in the medicine
bundle or the shrine or the museum, the holy
place, the area that contains what is sacred, and
then next day you probably do much the same
again if to do that is human, if that’s what it
takes, then I am a human being after all. Fully,
freely, gladly, for the first time.

Not, let it be said at once, an unaggressive or
uncombative human being. I am an aging, angry
woman laying mightily about me with my
handbag, fighting hoodlums off. However I don’t,
nor does anybody else, consider myself heroic
for doing so. It’s just one of those damned things
you have to do in order to be able to go on
gathering wild oats and telling stories.

It is the story that makes the difference. It is
the story that hid my humanity from me, the
story the mammoth hunters told about bashing,
thrusting, raping, killing, about the Hero. The
wonderful, poisonous story of Botulism. The
killer story.

It sometimes seems that the story is
approaching its end. Lest there be no more
telling of stories at all, some of us out here in the
wild oats, amid the alien corn, think we’d better
start telling another one, which maybe people
can go on with when the old one’s finished.
Maybe. The trouble is, we’ve all let ourselves
become part of the killer story, and so we may
get finished along with it. Hence it is with a
certain feeling of urgency that I seek the nature,
subject, words of the other story, the untold one,
the life story.

It’s unfamiliar, it doesn’t come easily,
thoughtlessly, to the lips as the killer story does;
but still, ''untold'' was an exaggeration. People 
have been telling the life story for ages, in all
sorts of words and ways. Myths of creation and
transformation, trickster stories, folktales, jokes,
novels...

The novel is a fundamentally unheroic kind of
story. Of course the Hero has frequently taken it
over, that being his imperial nature and
uncontrollable impulse, to take everything over
and run it while making stern decrees and laws
to control his uncontrollable impulse to kill it. So
the Hero has decreed through his mouthpieces
the Lawgivers, first, that the proper shape of the
narrative is that of the arrow or spear,
starting \textit{here} and going straight \textit{there} and THOK!
hitting its mark (which drops dead); second, that
the central concern of narrative, including the
novel, is conflict; and third, that the story isn’t
any good if he isn’t in it.

I differ with all of this. I would go so far as to
say that the natural, proper, fitting shape of the
novel might be that of a sack, a bag. A book holds
words. Words hold things. They bear meanings.
A novel is a medicine bundle, holding things in a
particular, powerful relation to one another and
to us.

One relationship among elements in the novel
may well be that of conflict, but the reduction of
narrative to conflict is absurd. (I have read a
how-to-write manual that said, ''A story should
be seen as a battle,'' and went on about strategies,
attacks, victory, etc.) Conflict, competition,
stress, struggle, etc., within the narrative

conceived as carrier
bag / belly / box / house / medicine bundle, may be
seen as necessary elements of a whole which
itself cannot be characterized either as conflict
or as harmony, since its purpose is neither
resolution nor stasis but continuing process.

Finally, it’s clear that the Hero does not look
well in this bag. He needs a stage or a pedestal or
a pinnacle. You put him in a bag and he looks like
a rabbit, like a potato.

That is why I like novels: instead of heroes
they have people in them.

So, when I came to write science-fiction
novels, I came lugging this great heavy sack of
stuff, my carrier bag full of wimps and klutzes,
and tiny grains of things smaller than a mustard
seed, and intricately woven nets which when
laboriously unknotted are seen to contain one
blue pebble, an imperturbably functioning
chronometer telling the time on another world,
and a mouse’s skull; full of beginnings without
ends, of initiations, of losses, of transformations
and translations, and far more tricks than
conflicts, far fewer triumphs than snares and
delusions; full of space ships that get stuck,
missions that fail, and people who don’t
understand. I said it was hard to make a gripping
tale of how we wrested the wild oats from their
husks, I didn’t say it was impossible. Who ever
said writing a novel was easy? If science fiction 
is the mythology of modern technology, then its
myth is tragic. 

''Technology'' or ''modern science'' (using the 
words as they are usually used, in an unexamined 
shorthand standing for the ''hard'' sciences and high
technology founded upon continuous economic
growth), is a heroic undertaking, Herculean,
Promethean, conceived as triumph, hence
ultimately as tragedy. The fiction embodying this
myth will be, and has been, triumphant (Man
conquers earth, space, aliens, death, the future,
etc.) and tragic (apocalypse, holocaust, then or
now).

If, however, one avoids the linear, progressive,
Time’s ---(killing)--- arrow mode of the TechnoHeroic,
and redefines technology and science as
primarily cultural carrier bag rather than weapon
of domination, one pleasant side effect is that
science fiction can be seen as a far less rigid,
narrow field, not necessarily Promethean or
apocalyptic at all, and in fact less a mythological
genre than a realistic one.

It is a strange realism, but it is a strange reality.

Science fiction properly conceived, like all
serious fiction, however funny, is a way of trying
to describe what is in fact going on, what people
actually do and feel, how people relate to
everything else in this vast stack, this belly of the
universe, this womb of things to be and tomb of
things that were, this unending story. In it, as in
all fiction, there is room enough to keep even
Man where he belongs, in his place in the scheme
of things; there is time enough to gather plenty of
wild oats and sow them too, and sing to little 

Oom, and listen to Ool’s joke, and watch newts,
and still the story isn’t over. Still there are seeds
to be gathered, and room in the bag of stars.














