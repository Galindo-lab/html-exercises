
En las regiones templadas y tropicales donde
parece que los homínidos evolucionaron a seres
humanos, la comida principal de la especie era
de origen vegetal. Del sesenta al ochenta por
ciento de lo que los seres humanos comían en
esas regiones en el periodo paleolítico, neolítico
y los tiempos prehistóricos, era principalmente
recolectado; solo en el extremo ártico la carne
era el alimento básico. Los cazadores de mamuts
espectacularmente ocupan la caverna y la mente,
pero lo que realmente hicimos para mantenernos
gordos y con vidas fue recolectar semillas, raíces,
germinados, tallos, hojas, nueces, bayas, frutas, y
granos, complementado de insectos y moluscos,
o atrapando pájaros, pescados, ratas, ratones, y
otros pequeños animales sin colmillos para
aumentar la proteína. Y ni siquiera trabajábamos
duro ---mucho menos que los campesinos
esclavizados en un campo ajeno después de que
la agricultura fuera inventada, mucho menos que
los trabajadores asalariados desde que la
civilización fue inventada. La persona prehistórica
promedio podía vivir cómodamente con una semana 
laboral de quince horas.

Quince horas a la semana para subsistir deja
mucho tiempo para otras cosas. Tanto tiempo
que tal vez los que no tenían un bebé alrededor
para animar sus vidas, o la habilidad de crear o
cocinar o cantar, o ideas muy interesantes para
pensar, decidieron hacerse a un lado e irse a
cazar mamuts. Los cazadores más hábiles
volverían con la sorpresa de una carga de carne,
un montón de marfil, y una historia. No era la
carne lo que hacía la diferencia. Era la historia.

Es difícil contar una historia cautivadora sobre
cómo arranqué una semilla de avena silvestre de
su cáscara, y luego otra, y luego otra, y luego otra,
y luego otra, y después de cómo me rasqué mis
picaduras de mosquitos, y que Ool mencionó
algo gracioso, y que fuimos al arroyo y tomamos
un sorbo, y que miramos unas salamandras por
un rato, y que encontramos otro brote de avena...
No, no se compara, no puede competir con cómo
atravesé mi lanza en lo profundo del inmenso
costado de un peludo animal, con cómo Oob fue
clavado en el movimiento de un colmillo, y cómo
se retorcía mientras gritaba, y cómo la sangre
salpicaba por todas partes en torrentes
carmesíes, y cómo Boob fue aplastado hasta
terminar en una masa gelatinosa cuando un
mamut cayó sobre él mientras yo disparaba mi
fecha infalible que justo atravesó por el ojo hasta
llegar al cerebro.

La historia no solo tiene Acción, tiene un Héroe.
Los Héroes son poderosos. Antes de que lo
sepas, los hombres y mujeres en el brote de
avena silvestre y sus niños, y sus habilidades para
crear, y los pensamientos del pensador, y las
canciones de los cantantes son todos parte de
ella, se han puesto a disposición de la historia del
Héroe. Pero no es la historia del grupo. Es su
historia.

Cuando estaba planeando el libro que terminó
como \textit{Three Guineas}, Virginia Woolf escribió
untitular en su libreta, ``Glosario''; ella había
pensado en reinventar el inglés de acuerdo con
su nuevo plan, con el fin de contar una historia
diferente. Una de las entradas de su glosario era
\textit{heroísmo}, definido como ``botulismo''. Y Héroe,
en el diccionario de Woolf, es ``botella''. El héroe
como botella, una reevaluación rigurosa. Ahora
propongo la botella como el héroe.

No solamente la botella de ginebra o vino, pero
la botella en su sentido más antiguo de un
contenedor en general, una cosa que guarda otra
más.

Si no consigues algo en donde poner cosas, la
comida se te va a escapar ---inclusive algo tan
inofensivo y sin utilidad como un grano de avena.
Pones tanto como pudieras en tu estómago
mientras sea práctico, pues este es el contenedor
principal; pero, ¿Qué pasa cuando a la mañana
siguiente despiertes y haga frío y se encuentre
lloviendo, no sería bueno tener un par de granos
de avena para poder comer y darle un poco a la
pequeña Oom para que se calle?, pero ¿Cómo le
harías para poder obtener más de un estómago
lleno y traer más de un puñado a casa? Así que te
levantas y vas al maldito brote de avena
empapado bajo la lluvia, ¿Y no sería bueno si
tuvieras algo en donde poner al bebé Oo Oo para
que pudieras recoger los granos de avena con las
dos manos? Una hoja, una vasija de calabaza, una
red, una cangurera, un saco, una botella, una
maceta, una caja, un contenedor. Un envase. Un
recipiente. O eso dice Elizabeth Fisher en Women’s
Creation (McCraw-Hill, 1975).

\begin{displayquote}
    \interlinepenalty=10000
    El primer artefacto cultural
    probablemente fue un recipiente...
    Muchos teóricos piensan que uno de los
    primeros inventos culturales tuvo que
    haber sido un contenedor para sostener
    las cosas recolectadas y algún tipo de
    cangurera o una red para cargar.
\end{displayquote}

Pero no, esto no puede ser. ¿Dónde está ese
maravilloso, grande, largo, objeto duro? Un hueso; 
creo, que el Hombre Primate primero golpeó a 
alguien en la película y después, lanzado gruñido hacia el
cielo, con éxtasis de haber cometido el primer
asesinato, dando vueltas ahí se convirtió en una
nave espacial impulsándose a través del cosmos
para fertilizarlo y producir al final de la película
un bonito feto, un varón obviamente,
desplazándose alrededor de la Vía Láctea
(curiosamente) ¿sin un vientre, sin ningún tipo de
matriz en lo absoluto? No lo sé. Ni me importa.
Yo no estoy contando esa historia. La hemos
escuchado, todos hemos escuchado acerca de
palos, lanzas y espadas, las cosas con que
golpear, atacar y pegar, las cosas largas y duras,
pero no hemos escuchado sobre las cosas en
donde ponemos cosas, el contenedor para que
las cosas sean guardadas. Eso es una nueva
historia. Eso son noticias.

herramientas para amartillar, triturar y cavar 
---cuál sería el uso de escarbar un montón de papas
si no tienes dónde cargar las que no te puedes
comer en casa--- o antes de la herramienta que
externaliza la energía, creamos la herramienta
que trae la energía a casa. Para mí tiene sentido.
Soy una seguidora a lo que Fisher llama ``La
teoría de la mochila de carga de la evolución
humana''.

Esta teoría no solo explica grandes áreas de
oscurantismo teórico, sino que también evita
grandes áreas de sinsentido teórico (habitadas
ampliamente por tigres, zorros, y otros animales
altamente territoriales). También me aterriza,
personalmente, en la cultura humana de una
forma en la que nunca me había sentido
aterrizada anteriormente. Mientras la cultura sea
explicada, en que se originó a partir de la
elaboración y el uso de objetos, largos y duros
para clavar, golpear y asesinar, nunca me había
visualizado en querer alguna participación en
esta. (``Lo que Freud malentendió por falta de
civilización de la mujer, es realmente su falta de
lealtad a la civilización'' Lilian Smith observó). La
sociedad, la civilización de la que hablan estos
teóricos, es evidentemente de ellos, son dueños
de esta, les gusta; fueron humanos,
completamente humanos, golpeando, picando,
clavando, matando. También yo queriendo ser
un humano, buscaba evidencia de que lo era,
pero si eso lo que hacía falta, hacer un arma para
matar con esta, entonces evidentemente yo era o
muy defectuosa como ser humano o para nada uno.

Eso es correcto, dijeron. Lo que tú eres, es una
mujer. Posiblemente nada que ver con un
humano, obviamente defectuosa. Ahora por
favor guarda silencio mientras contamos la
Historia del ascenso del Hombre como Héroe.

Continúen, digo yo, mientras camino hacia las
plantas de avena silvestre, junto con Oo Oo en la
cangurera y el pequeño Oom cargando la
canasta. Continúen narrando cómo el mamut
cayó sobre Boob y cómo Caín cayó en Abel y
cómo la bomba cayó en Nagasaki y cómo la masa
en fuego cayó sobre los habitantes y cómo los
misiles caerán sobre el Imperio Malvado, y los
otros pasos hacia el Ascenso del Hombre.

Si es algo humano hacer, poner, algo que tú
quieres, porque es útil, comestible, o hermoso,
en una bolsa, o una canasta, o un poco de corteza
u hoja enrollada, o una red tejida de tu propio
cabello, o lo que tengas, y que luego lo lleves a
casa contigo, la casa siendo otro tipo de bolsa o
saco, un contenedor para personas, y después tú
lo sacas y comes o lo compartes o lo guardas para
un lugar sagrado, el área que contiene lo que es
sagrado, y probablemente al día siguiente
termines haciendo lo mismo ---si hacer eso es
humano, si eso es lo necesario, entonces al final
soy un ser humano después de todo. Completa,
libre, con gusto, por la primera vez.

No, que se diga de una vez, un ser humano pasivo
y sin disposición a pelear. Soy una mujer mayor
enojada definiéndome poderosamente con mi
bolsa, ahuyentando a rufianes. Pero ni yo, ni
nadie más, se considera como alguien heroico
por hacerlo. Solo es una de las malditas cosas que
tienes que hacer para poder ir a recolectar
semillas de avena silvestre o contar historias.

Es la historia lo que hace la diferencia. Es la
historia que esconde mi humanidad de mí misma.
La historia que los cazadores de mamuts cuentan
del Héroe sobre golpear, clavar, violar, matar. La
maravillosa historia venenosa del Botulismo. La
historia asesina.

A veces puede parecer que la historia está
acercándose a su final. Pero, para que ya no se
cuenten más historias, algunos de nosotros aquí
entre las plantas de avena silvestre, entre el maíz
ajeno, pensamos que sería mejor si empezáramos
a contar otra historia, a la cual la gente pueda ir
cuando la vieja historia haya terminado. Talvez.
El problema es, que nos hemos permitido ser
parte de la historia de los asesinos, así que
podríamos acabar junto con ella. Por lo tanto,
con un cierto sentido de urgencia, busco la
naturaleza, el tema, las palabras de la otra
historia, la no narrada, la historia de la vida.

Resulta desconocido, no llega con facilidad a los
labios, sin pensar, como la historia de los asesinos
lo hace; pero, aun así, no narrada fue una
exageración. La gente ha contado la historia de
la vida por siglos, en todo tipo de formas y
palabras. Mitos de la creación y la
transformación, Historias de timadores, cuentos
populares, chistes, novelas...

La novela fundamentalmente es un tipo de
historia no heroica. Obviamente el Héroe de
manera frecuente ha tomado posesión sobre
esta, siendo su naturaleza imperial y con un
impulso incontrolable, de tomar posesión sobre
todo y manejarlo mediante la toma de decisiones
tercas y leyes para controlar su incontrolable
impulso de matarla. Así que el Héroe ha de
decretar por medio de sus portavoces, los
legisladores, por lo tanto, pone en primer lugar,
que la forma correcta de la narrativa sea la de una
lanza, empezando aquí y yendo directamente ahí
y ¡THOK! Acertando directo al objetivo (que cae
muerto); y, en segundo lugar, que la intriga
central de la narrativa, incluyendo la novela, sea
el conflicto, y por tercero, que la historia no sea
buena si él no está en ella. 

Difiero con todo esto. Yo iría tan lejos como
decir que la figura propia, natural, de la forma de
la novela puede ser la de un saco, una bolsa. Un
libro contiene palabras.  Contienen un
significado. Una novela es un paquete sagrado \footnote{ 
    \interlinepenalty=10000
    Nota de la traducción. En el texto original se utiliza
    la palabra \textit{healing bundle}, concepto proveniente de
    las culturas nativo americanas y mesoamericanas,
    que hace referencia a un conjunto de objetos
    sagrados envueltos. En el idioma náhuatl también
    existe la palabra \textit{tlaquimilolli} que tiene una función
    similar. A grandes rasgos se pueden describir como
    contenedores de la fuerza divina, que servían como
    símbolos de la identidad del grupo.
}

que se encuentra sosteniendo los elementos en
una relación particular, poderosa, entre estas y
nosotros.

Una relación entre los elementos de una novela
que bien puede ser el conflicto, pero la
reducción de la novela al conflicto es absurdo.
(He leído un manual de ``como escribir'' que
decía, ``Una historia debe ser vista como una
batalla'', y continuaba sobre estrategias, ataques,
victorias, etc.) Conflicto, competencia, estrés,
lucha, etc., concebidas dentro de la narrativa
tomada como un saco / cangurera / mochila / 
caja / casa / paquete
sagrado, pueden ser vistos como elementos
necesarios de un total que no puede ser
clasificado ni como conflicto ni como harmonía,
ya que el propósito no es la resolución o la
estasis, sino, un proceso continuo.

Finalmente está claro que el Héroe no observa
claramente su bolsa. Él necesita un escenario o
un pedestal o un pináculo. Lo pones en una bolsa
y se parece a un conejo, a una papa.

Esto es lo que me gusta de las novelas: en lugar
de héroes tienen personas.

Así que, cuando se me presente el escribir
novelas de ciencia ficción, llego cargando este
gran saco lleno de cosas, mi mochila llena de mis
debilidades y rarezas, y pequeños fragmentos de
cosas más pequeñas que una semilla de mostaza,
e intrincadas redes tejidas que cuando
laboriosamente se desenredan, te encuentras
una piedrita azul, un imperturbable cronómetro 
funcionando que me dice la hora de otro mundo,
el cráneo de un ratón; lleno de inicios sin fin, de
iniciaciones, de pérdidas, de transformaciones y
traducciones, y muchos más trucos que
conflictos, muchos menos triunfos que trampas y
engaños; llena de naves que quedan atoradas,
misiones que fallan, y gente que no entiende.
Digo que es difícil hacer una historia que atrape,
que se trate de cómo arrancamos las semillas de
avena silvestre de su cáscara, pero no dije que
fuera imposible. ¿Quién dijo que escribir una
novela era sencillo?

Si la ciencia ficción es la mitología de la
tecnología moderna, entonces su mito es trágico.
La ``tecnología'', o la ``ciencia moderna'' (usando
las palabras como habitualmente son usadas, sin
examinar y de una forma sesgada, que entiende
por las ciencias ``exactas'' y la alta tecnología a un
continuo crecimiento económico) es una tarea
heroica, Herculina, Prometeica, concebida como
un triunfo, por lo tanto, en última instancia, como
tragedia. La ficción que representa este mito
será, y ha sido, triunfante (El hombre conquista
la tierra, el espacio, los alienígenas, la muerte, el
futuro, etc.) y trágica (apocalipsis, holocausto,
entonces y ahora).

Pero, si uno evade el modo lineal, progresivo, del
tiempo en forma de flecha (asesino) de lo TecnoHeroico, y
redefine la tecnología y ciencia como
las principales bolsas de carga culturales, en lugar
del arma de dominación, encontramos el lado
placentero de la ciencia ficción, que puede ser
vista como una forma menos rígida, menos estrecha, 
no necesariamente prometeica o en lo
absoluto apocalíptica, de hecho, resulta menos
en un género mitológico que en uno realista.

Es un realismo extraño, pero es una extraña
realidad.

La ciencia ficción concebida de una forma
apropiada, como toda la ficción sería, por más
gracioso, una manera de describir lo que está
sucediendo, de lo que la gente hace y siente, de
cómo la gente se relaciona a lo demás en este
inmenso conjunto, en este ombligo del universo,
en este vientre de cosas por ser y tumbas de las
cosas que fueron, esta historia sin fin. En ella,
como en toda ficción, hay espacio suficiente para
mantener incluso al Hombre donde pertenece,
en su lugar en el esquema de las cosas; hay
tiempo suficiente para recolectar suficientes
semillas de avena silvestre y también para
sembrarlas, y cantarle al pequeño Oom, y
escuchar los chistes de Ool, y mirar a las
salamandras, y aun así, la historia no acabaría.
Aún hay semillas que recolectar, y lugar en
nuestras bolsas para las estrellas.